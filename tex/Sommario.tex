\begin{center}
	\subsubsection{Sommario}
\end{center}

		In questo articolo ho deciso di effettuare un approfondimento circa le firme digitali, un argomento secondo me molto interessante e ricorrente, in quanto stiamo andando incontro ad una forte e veloce digitalizzazione. Il contenuto del documento, oltre ad osservare in maniera approfondita le modalità, le tecniche e gli algoritmi utilizzati per la corretta apposizione della firma, vuole mostrare l'impatto globale della tecnologia citata. Nella parte introduttiva verrà illustrato il funzionamento generale e definiti i concetti principali. Terminata l'introduzione, si potranno osservare gli algoritmi utilizzati per l'apposizione e verifica della firma, in ordine, in funzione della loro importanza e qualità. Per concludere, è stato dedicato uno spazio non indifferente circa le possibili vulnerabilità dei più noti algoritmi di firma digitale. Prima di procedere con la lettura, credo sia interessante conoscere le statistiche più significative legate al fenomeno che verrà trattato, in quanto permettono di tracciare un quadro generale:
		\\
		\begin{itemize}
			\item la fetta di mercato conquistata dalle firme digitali stimata nel 2020 è di 3.56 miliardi di dollari, si stima che nel 2030 raggiungerà quota 61.91, la pandemia ha rivestito un ruolo non indifferente in questa crescita;
			\item le aziende che decidono di abbandonare le firme manuali risparmiano l'80 \% dei costi di consegna;
			\item la digitalizzazione delle firme, riduce notevolmente la presenza di errori, fino a raggiungere i 90 \% in più di efficienza;
			\item il 65 \% delle aziende che sfruttano i documenti cartacei, sprecano in media un intero giorno di lavoro in più rispetto a opta per il digitale.
		\end{itemize}