\chapter{L'algoritmo EdDSA}

L'algoritmo ECDSA osservato nel capitolo precedente è particolarmente efficace e garantisce un livello di sicurezza non indifferente, detto ciò, è possibile analizzare soluzioni, sotto certi punti di vista, migliori. Mi sto riferendo all'algoritmo EdDSA  (anch'esso basato sull'utilizzo di curve ellittiche), simile per molti aspetti all'ECDSA, con la differenza che, il primo è più semplice da comprendere e da implementare. Inoltre EdDSA, rispetto alle curve più popolari (ad esempio edwards448) è più performante rispetto al precedentemente citato (quindi dipende molto dalle curve usate e dalla specifica implementazione). Infine è importante sapere che, EdDSA non mette a disposizione una tecnica per il recupero della chiave pubblica della firma e del messaggio, a differenza del precedente. Le varianti più comuni sono lo Ed25519 ed Ed448, che sono descritte nello standard RFC 8032.

\section{Generazione coppia di chiavi}

Entrambe le versioni dell'algoritmo citate, generano chiavi e firme di piccole dimensioni, garantendo allo stesso tempo un alto livello di sicurezza. Supponendo che la curva ellittica che verrà utilizzata comprenda un punto generatore $G$ e un sottogruppo di ordine $q$, generato a partire da $G$, abbiamo che la coppia di chiavi è composta da:

\begin{itemize}
	\item chiave privata $\rightarrow$ intero che viene generato a partire da un seed casuale (di grandezza simile a $q$), che verrà hashato e alla quale verranno applicate diverse trasformazioni, che garantiscano che la chiave privata appartenga sempre allo stesso sottogruppo di punti di curva ellittica e che le chiavi abbiano sempre una lunghezza in bit simile (per proteggersi dagli attacchi di canale laterale basati sulla temporizzazione). La grandezza della chiave privata varia in funzione della variante dell'algoritmo sfruttata;
	\item chiave pubblica $\rightarrow$ chiave privata $\cdot$ G, viene codificata e compressa in un punto di curva ellittica.
\end{itemize}

\subsubsection{Implementazione in python}

In questo capitolo, verrà mostrata l'implementazione pratica della versione ed25519.

\begin{lstlisting}
	import ed25519
	
	privKey, pubKey = ed25519.create_keypair()
\end{lstlisting}

\section{Generazione della firma}

La firma viene generata a partire da un messaggio di testo e dalla chiave privata del firmatario, producendo in output la firma, composta dalla coppia di interi $\{R, s\}$.
\begin{center}
	$f(msg, privateKey) \rightarrow \{R, s\}$
\end{center}
I passaggi da seguire per la corretta generazione della firma sono i seguenti:

Deterministically generate a secret integer r = hash(hash(privKey) + msg) mod q (this is a bit simplified

\begin{itemize}
	\item $r = hash(hash(privateKey) + msg) mod q$ $\rightarrow$ al primo passaggio viene generato l'intero $r$, che rappresenterà l'input del prossimo passaggio;
	\item $R = r * G$ $\rightarrow$ successivamente calcolo $R$ (il punto in cui si trova la chiave pubblica), primo valore della coppia che costituisce la firma, moltiplicando $r$ per il punto generatore $G$; 
	\item $h = hash(R + pubKey + msg) mod q$ $\rightarrow$ a questo punto si calcola l'hash a partire da $R$, chiave pubblica e messaggio;
	\item $s = (r + h * privKey) mod q$ $\rightarrow$ infine viene calcolato $s$.
\end{itemize}

Al termine di questi passaggi è possibile ritornare la firma $\{R, s\}$.

\subsubsection{Implementazione in python}

La coppia di chiavi del Ed25519 è generato in maniera random, prima viene creato un seme di 32 byte, dalla quale viene creata una chiave privata. Dalla chiave privata poi viene generata quella pubblica, lo hash sfruttato è lo SHA-512.

\begin{lstlisting}
msg = 'messaggio di prova'
signature = privKey.sign(msg, encoding='hex')
\end{lstlisting}

\section{Verifica}

La verifica della firma richiede in input un messaggio di testo, la coppia di chiavi del firmatario e produce in output un valore booleano, che se vero ne dimostra la validità. 

La verifica si effettua nei seguenti passaggi:

\begin{itemize}
	\item $h = hash(R + pubKey + msg) mod q$ $\rightarrow$ calcolo dello hash, similmente a quello ottenuto nella fase di firma;
	\item $P_1 = s \cdot G$ $\rightarrow$;
	\item $P_2 = R + (h \cdot privateKey)$;
\end{itemize}

Se $P_1$ risulta uguale a $P_2$ la verifica della firma è andata a buon fine.
Questo perché durante la firma, abbiamo $s = (r + h \cdot privateKey) \: mod \: q$ e sostituendo s nell'equazione di $P_1$ si ottiene:
\begin{center}
	$P1 = s \cdot G = (r + h \cdot privateKey) mod q \cdot G = r \cdot G + h \cdot privKey \cdot G = R + h \cdot pubKey$
\end{center}

Quanto sopra è esattamente l'altro punto $P_2$, se i punti $P_1$ e $P_2$ sono lo stesso punto di curva ellittica, ciò dimostra che il punto $P_1$, calcolato dalla chiave privata, corrisponde al punto $P_2$, creato dalla sua corrispondente chiave pubblica.

\subsubsection{Implementazione in python}

\begin{lstlisting}
	try:
		pubKey.verify(signature, msg, encoding='hex')
		print("The signature is valid.")
	except:
		print("Invalid signature!")
\end{lstlisting}